\documentclass{twocolcv}

\usepackage{lipsum}

\begin{document}
\header{Jonatan Pålsson}{Jonatan-crop.png}{YellowGreen}
\section*{Personal data}
  \cvtable{
    Name           & Jonatan Pålsson                 \\
    Phone number   & +46 (0)73 698 7205             \\
    E-mail / jabber& jonatan.p@gmail.com    \\
    Address        & Göteborg, Sweden     \\
    Date of birth  & 1989 01 06
  }
\section*{Academical merits}
  \cvtable{
    \emph{2011-2013} & \uniline{MSc.} {Computer Science, Algorithms, Languages and Logic} {Chalmers University} {Johanneberg Campus}
    \emph{2008-2011} & \uniline{BSc.} {Computer Science} {University of Gothenburg} {Johanneberg Campus}
  }

\section*{Master's Thesis}
This thesis investigates the required properties and components of a low latency multimedia indexing system for automotive use. A survey of existing systems was made, and the Tracker indexing system was found to be most suitable. Tracker was then modified and improved to meet the automotive requirements set by Pelagicore, who also requested the thesis.

\noindent Available here: \url{https://github.com/jonte/theses/raw/master/msc_thesis.pdf}
\section*{Bachelor's Thesis}
The thesis investigates whether it is possible to create a programming language agnostic application server specialized for game-like applications. The project was carried out by four team members and I learned many things about team work and organization, in addition to the actual technical aspects of the project. A prototype capable of powering several hundred simple game instances written in JavaScript was developed in Erlang.

\noindent Available here: \url{https://github.com/jonte/theses/raw/master/bsc_thesis.pdf}
\section*{Professional merits}
\subsection*{Software Engineer}
\emph{Pelagicore}\\
June 2013 -- Ongoing \newline\newline
I've worked with media indexing, process containment, BSP development, Qt development, Multimedia playback (mainly GStreamer) and some HMI development. Peagicore develops embedded Linux systems for Automotive IVI systems, so I have done a lot of embedded Linux programming as well. I am also a member of two GENIVI expert groups; the Consumer Electronics \& Connectivity Expert Group and the Networking Expert Group. I have given talks at FOSDEM and am active in open source development. I have been the lead engineer for the GENIVI Media Manager project, spanning approximately half a year.

\subsection*{Thesis worker}
\emph{Pelagicore}\\
January 2013 -- June 2013 \newline\newline
I wrote my Master's thesis in the Gothenburg office of Pelagicore, see Master's thesis section.

\subsubsection*{Software developer}
\emph{Mezmerize-e} \\
February 2011 -- February 2012\newline\newline
During my time here, I developed a product recommendation engine for e-commerce. The engine suggested products based on purchase patterns and product similarity. I worked with Python, PHP, memcached, Apache and MySQL.

\section*{Spoken languages}
  \cvtable{
    Swedish     & I have lived in Sweden for all my life and I speak Swedish natively \\\\
    English     & I speak English fluently
  }

\section*{Technical expertise}
\subsubsection*{Programming languages}
I can work with the following languages: C, C++, Python, Haskell, Erlang, C\#, Java, PHP, JavaScript, CSS, HTML, MySQL, \LaTeX
\subsubsection*{Operating systems}
I am proficient in the following operating systems: Desktop/server Linuxes (mainly Debian \& derivatives, Arch Linux), Windows XP -- 8. Greatly prefer Debian.
\subsubsection*{Development environments \& IDEs}
I am proficient in Vim, Microsoft Visual Studio (for C\#), and Eclipse (for Java)
\subsubsection*{Tools \& Frameworks}
I have experience in Qt, Glib and Glibmm. I have experience in building embedded Linux distributions using Yocto. I have experience in using (not porting) U-Boot on ARM. I have developed software for Tizen-IVI, and am comfortable using GBS.

\section*{Personal notes}
I love technology and software development. I have strong opinions on software and systems, because these things are important to me. I am an avid Linux user, and I really, really like UNIX-like systems. I enjoy working in open source communities, and I strongly empathize with open source ideas and ideals, software wants to be free :)

For me, my job needs to be challenging, and my colleagues need to be competent, which means we can learn from each other. Learning new things is very important, and I enjoy keeping up to date with the latest news in the projects I enjoy. I also enjoy teaching co-workers, or community members new things in order to build better projects.
\end{document}
