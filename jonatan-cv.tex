\documentclass{twocolcv}

\usepackage{lipsum}

\begin{document}
\header{Jonatan Pålsson}{Jonatan-crop.png}{YellowGreen}
\section*{Personal data}
  \cvtable{
    Name           & Jonatan Pålsson    \\
    Phone number   & +46 (0)73 698 7205 \\
    E-mail         & jonatan@jptk.se 	\\
    Address        & Göteborg, Sweden   \\
    Date of birth  & 1989 01 06
  }
\section*{Professional merits}
\employment{JP Teknikkonsult AB}
           {Systems and Software engineering}
           {February 2020 -- Ongoing}
           {
\project{Embedded linux consulting}
        {As an independent contractor I have provided expertise in large embedded Linux systems based on custom Yocto configurations. In this role I have implemented software, as well as provided teaching and coaching in developing modern Linux systems and using modern software engineering methodologies.}
}
\employment{Luxoft Sweden AB}
           {Senior Software Engineer}
           {October 2016 -- February 2020}
           {
\project{Technical team lead}
        {I have lead a Scrum team of seven engineers in developing and maintaining an automotive Linux distribution. The distribution was developed using the Yocto project. The team was responsible for structuring the platform and workflows around it in a logical way, and maintaining it as software was integrated by development teams (which included our own team). The platform was successfully used on five different hardware targets, by more than 100 developers.}
\newline\newline
\project{Technical team lead}
        {I have lead a Scrum team of seven engineers in designing and implementing automotive middleware software in C++. This task involved designing software and APIs, analyzing requirements, component implementation in C++, as well as implementing tooling in Python.
}
}
\employment{Pelagicore}
           {Senior Software Engineer}
           {June 2015 -- October 2016}
           {
I have taken lead in multiple projects, below are some samples.\newline

\project{Yocto-based Linux platforms}
        {I have developed several automotive development platforms for car manufacturers. These platforms have been developed together with the car manufacturers, according to their specific requirements. Typical use-cases for these platforms has been to enable middleware, and UI development by other developers, and to serve as demonstrators. The platforms have been Yocto-based (typically on the Poky distribution). I have tailored the platforms to customer demands regarding upstart (such as customization of systemd and systemd units), kernel modifications (including development of device drivers), hardware support (WiFi, Bluetooth, graphics drivers, bootloader customizations, etc.). Most of these projects have used NVIDIA hardware.}
\newline\newline
\project{Android Auto demonstrator}
        {I developed an Android Auto demonstrator capable of audio and video streaming, as well as providing a basic command channel. The demonstrator had a Qt UI which displayed a video stream rendered using GStreamer, captured audio input using PulseAudio and provided basic audio and video focus handling. The platform itself was built using Yocto, based on Poky and an automotive-specific layer for infotainment systems (meta-ivi). I gave a talk describing the implementation of the project on the GENIVI All Members Meeting.}
\newline\newline
\project{Radio tuner stack}
        {I developed a complete solution for tuning AM, FM, and DAB radio. In the project I developed Linux device drivers for different tuner and DSP chips, wrote an Android native service to go on top on the device drivers, and an Android application allowing the user to interface with the stack. The software was layered in such a way that I could replace the Android Binder-based service with a D-Bus service, and run most of the code on vanilla Linux. Besides controlling the hardware, I also implemented support for radio standards such as RDS and TMC for FM, and MOT for DAB, this was done without third-party libraries.}
}

\employment{Pelagicore}
     {Software engineer}
     {June 2013 -- 2015}
     {I've worked with media indexing, process containment, BSP development, Qt development, Multimedia playback (mainly GStreamer) and some HMI development. Pelagicore develops embedded Linux systems for Automotive IVI systems, so I have done a lot of embedded Linux programming as well. I was also a member of two GENIVI expert groups; the Consumer Electronics \& Connectivity Expert Group and the Networking Expert Group. I have given talks at FOSDEM and am active in open source development. I have been the lead engineer for the GENIVI Media Manager project, spanning approximately half a year.}

\employment{Pelagicore}
     {Thesis worker}
     {January 2013 -- June 2013}
     {I wrote my Master's thesis in the Gothenburg office of Pelagicore, see Master's thesis section.}

\employment{Mezmerize-e}
           {Software developer}
           {February 2011 -- February 2012}
           {During my time here, I developed a product recommendation engine for e-commerce. The engine suggested products based on purchase patterns and product similarity. I worked with Python, PHP, memcached, Apache and MySQL.}

\section*{Spoken languages}
  \cvtable{
    Swedish     & I have lived in Sweden for all my life and I speak Swedish natively \\\\
    English     & I speak English fluently
  }

\section*{Technical expertise}
\bulletlist{Programming languages}{
    \bitem {C}      {1}{Written several large systems with C}
    \bitem {C++}    {2}{Written several medium sized programs}
    \bitem {Python} {2}{Used for systems testing and smaller applications}
    \bitem {Bash}   {2}{Developed and maintained large bash scripts for legacy systems}
    \bitem {QML}    {3}{Done some QML/C++ apps, written some QML C++ plugins and models}
    \bitem {Haskell}{4}{Used at university courses and for side projects}
    \bitem {Erlang} {4}{Used for BSc. thesis and university courses}
    \bitem {C\#}    {4}{Used in MOOCs}
    \bitem {Java}   {4}{Used in university courses}
    \bitem {\LaTeX} {4}{Used for university reports and for fun}
    \bitem {rust}   {4}{Used for fun}
    \bitem {PHP, JavaScript, CSS, HTML, MySQL}{4}{Used in previous employment}
}

In general, I think programming languages are fun, and I enjoy learning new ones.

\bulletlist{Operating systems}{
    \bitem {Desktop/server Linuxes} {1}{Greatly prefer Debian. Used Linux for more than a decade}
    \bitem {WindRiver Linux}        {2}{I have provided expert support for WRL5/8/18 on a contractor basis}
    \bitem {Windows XP -- 10}       {2}{Used at home from time to time}
}
\bulletlist{Development environments \& IDEs}{
    \bitem {Vim}            {1} {Environment of choice.}
    \bitem {Eclipse}        {3} {Used for Java development}
    \bitem {Android Studio} {3} {Used for Java Android apps}
    \bitem {Visual Studio}  {4} {Used for C\# development}
}

\bulletlist{Tools, Frameworks \& Tech} {
    \bitem {Git}            {1}{I have implemented my own git tool in C :-)}
    \bitem {Yocto}          {1}{Written recipes, made \& maintained layers, and provided expert consultancy services}
    \bitem {Android Binder} {2}{Developed several reasonably complex native Android services and sandboxed apps which have used the Binder IPC}
    \bitem {Linux kernel}   {2}{Written several GPIO, I$^2$C and SPI device drivers. Configured device trees, optimized kernels for boot speed}
    \bitem {Gerrit}         {2}{I have worked with Gerrit in several projects}
    \bitem {Jenkins}        {3}{Used and configured professionally for contiguous integration }
    \bitem {Travis-CI}      {3}{Used and configured personally for automatic build and pull request analysis }
    \bitem {Qt Core}        {3}{Used for medium sized projects}
    \bitem {Glib}           {3}{Used for medium sized projects}
    \bitem {Glibmm}         {3}{Used for medium sized projects}
    \bitem {GNU Make}       {3}{Developed several reasonably complex build scaffoldings}
    \bitem {CMake}          {3}{Build system of choice, developed several reasonably complex build scaffoldings}
    \bitem {U-Boot}         {3} {Used on ARM targets}
    \bitem {Tizen-IVI}      {3} {Developed some software for the platform}
    \bitem {Process containment}    {3} {Co-authoring a book on cgroups and namespaces \footnote{\url{https://github.com/containerbook/containerbook/}}}
    \bitem {Android build system}   {4} {Developed several Android native and sandboxed apps/services in AOSP}
    \bitem {Git Build System (GBS)} {4} {Built Tizen-IVI and software for it using GBS}
}

\bulletlist{Chips \& hardware} {
    \bitem {NXP SAF775}  {3}{I have written device drivers and middleware software to interact with the SAF775 (Dirana3)}
    \bitem {NXP SAF36}   {3}{I have written device drivers and middleware software to interact with the SAF36 (SATURN)}
    \bitem {NXP TEF701}  {3}{I have written device drivers and middleware software to interact with the TEF701 (SABRE)}
    \bitem {1-Wire} {3}{I have developed QEMU device support for 1-Wire, as well as developed custom 1-Wire hardware for DS2482-800.}
    \bitem {PCI Express} {4}{I have developed device drivers for PCI-Express devices, mainly for multi-functional devices where the PCI-E link has been used to multiplex other busses}
    \bitem {SPI}         {4}{I have developed device driver for SPI devices, mainly for radio tuner hardware}
    \bitem {I$^2$C}      {4}{I have developed device driver for SPI devices, mainly for radio tuner hardware}
    \bitem {CAN}         {4}{I have developed userspace code for interacting with CAN devices in automotive networks}
}

\section*{Academical merits}
  \cvtable{
    \emph{2011-2013} & \uniline{MSc.} {Computer Science, Algorithms, Languages and Logic} {Chalmers University} {Johanneberg Campus}
    \emph{2008-2011} & \uniline{BSc.} {Computer Science} {University of Gothenburg} {Johanneberg Campus}
  }

  \subsection*{Master's Thesis}
This thesis\footnote{\url{https://github.com/jonte/theses/raw/master/msc_thesis.pdf}} investigates the required properties and components of a low latency multimedia indexing system for automotive use. A survey of existing systems was made, and the Tracker indexing system was found to be most suitable. Tracker was then modified and improved to meet the automotive requirements set by Pelagicore, who also requested the thesis.

\subsection*{Bachelor's Thesis}
The thesis\footnote{\url{https://github.com/jonte/theses/raw/master/bsc_thesis.pdf}} investigates whether it is possible to create a programming language agnostic application server specialized for game-like applications. The project was carried out by four team members and I learned many things about team work and organization, in addition to the actual technical aspects of the project. A prototype capable of powering several hundred simple game instances written in JavaScript was developed in Erlang.

\section*{Personal notes}
I enjoy working in open source communities, and I strongly empathize with open source ideas and ideals. Software wants to be free :). My current focus is independent contracting, since this allows me to select the projects I find most interesting. My current interests are embedded software in general, and Linux in particular.

For me, my job needs to be challenging, and my colleagues need to be competent, which means we can learn from each other. Learning new things is very important, and I enjoy keeping up to date with the latest news in the projects I enjoy. I also enjoy teaching co-workers, or community members new things in order to build better projects.
\end{document}
